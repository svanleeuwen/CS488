\documentclass {article}
\usepackage{fullpage}
\usepackage{enumerate}
\usepackage{cite}

\begin{document}

~\vfill
\begin{center}
\Large

A5 Project Proposal

Title: Ray Traced Tetris

Name: Spencer Van Leeuwen

Student ID: 20412199

User ID: srvanlee
\end{center}
\vfill ~\vfill~
\newpage
\noindent{\Large \bf Final Project:}
\begin{description}
\item[Purpose]:\\
    The purpose of my project is to build a real-time Whitted raytracer. As objects
    in the scene move, the raytracer will be able to re-render the scene in real-time.


\item[Statement]:\\
    As hardware gets better, more and more ray tracing researchers have become
    interested in the idea of trying to build real-time raytracers, since there
    are things that can be done much better using ray tracing than rasterization
    (the most simple examples are reflection and refraction). \\

    \cite{CGF13} is an article overviewing different methods of ray tracing animated
    scenes, and is where I found most of the other papers that I am using to build my
    project. The foundation of this project will be quickly partitioning the objects in the
    scene and tracing multiple rays. This should reduce the rendering time enough
    to be able to create an animated image. \\

    For my final scene, I hoping to make a raytraced tetris game (if I can't
    get enough FPS, I might come up with another idea for the revised proposal). The pieces
    will have different materials, instead of just different colours. I also might
    experiment with putting other primitives in the scene, like a mirror behind the
    game. It would also be interesting to make all of the pieces translucent and having
    an object behind the game to demonstrate refraction. The border of the game and the
    surface that the game sits on (probably just a plane), will have an interesting texture
    of some sort. \\

    This project is very interesting because it is based on research that requires very
    innovative solutions to work. Real-time raytracers don't have access to special purpose
    hardware, unlike rasterization which has highly parallelized GPUs. Plus, finding ways
    to make something run faster is just inherently interesting to any computer scientist. \\

	I will learn how to add basic improvements to a raytracer, like reflection, refraction,
    antialiasing, and texturing.
    I will also learn about fast building and traversal of spacial partitioning data structures.

\item[Technical Outline]:

\subsubsection*{Bounding Interval Hierarchy} 
    This data structure was recommended in \cite{CGF13} for being fast and relatively
    easy to implement. The BIH is introduced in \cite{Wa06}, and was significantly faster
    than any previous data structure. \\
    
    When constructing a BIH, it is like a hybrid of a kd-tree and a Bounding Volume Hierarchy. 
    First, the global bounding box is divided into a uniform grid. At each step, we choose
    the longest axis with respect to the uniform bounding box and get the median along that axis. 
    Then we partition the objects in the bounding box to below and above the median, by checking
    which side contains more of the object's bounding box. 
    Then for the objects below the median, we take the plane that bisects
    the median and increase its location on the chosen axis such that all of the objects' bounding
    boxes are completely below the plane. We can induce a new bounding box using this plane.
    We can do a similar thing for objects above the median. Again, the paper \cite{Wa06} 
    explains this much better. \\

    To traverse the BIH, we simply treat it as a BVH, where the bounding volumes are induced
    by the planes we found during the construction. \\

    To demonstrate that this works, I will time the rendering of a few different scenes
    with and without the data structure.

\subsubsection*{Reflection Rays}
    I will just use a simple reflection model, where $\theta_{in} = \theta_{out}$, with
    attenuation per reflection. I might reference \cite{PG10} to figure out a nice way
    to do it.

\subsubsection*{Refraction Rays}
    I am just going to use a basic refraction model. I will use \cite{PG10} to figure
    this out.

\subsubsection*{Antialiasing}
    I am going to use Stratified sampling, as described in \cite{PG10}. Basically, I am
    going to split each pixel into a grid, then take a random point in each grid box
    and sample there.

\subsubsection*{Ray Packets}
    For ray-packet tracing, the basic idea is to group a bunch of similar rays together
    at test if they intersect a primitive all at once. \cite{BEL07} shows that it is
    possible to use ray packets for secondary rays, even though they don't share
    an origin. It suggests using the BVH traversal method for ray packets described in
    \cite{WBS07}. When testing against a particular primitive, \cite{BEL07} suggests
    using interval arithmetic, as described in \cite{BWS06}. \\

    The traversal as described in \cite{WBS07} has a few key components. First, it
    tests if the first ray in the packet hits the bounding box. If it does, we recurse
    the children. Otherwise, we perform an "all miss" test, which conservatively tests
    if all of the rays in the packet miss the bounding box. The paper actually shows that
    in the majority of cases, one of the tests succeed and saves a ton of computation. \\

    Again, I plan on using a table with times to demonstrate that this works.

\subsubsection*{Add Tetris Primitive}
    I am going to add the tetris game from A1 as a lua primitive. This will allow me to place
    the tetris game where I want in the scene. At first, this will just place
    the border in the scene, but as the game progresses, blocks will be added as children.
    To make sure this works, I will have to update the game a few times before drawing,
    just so some blocks are visible.

\subsubsection*{Animate Tetris using Qt}
    I am going to set up Qt so that after rendering an image, it displays on the canvas.
    This is tied in with the previous objective, since I don't actually have to 
    support the blocks moving until this point.

\subsubsection*{Manage user input using Qt}
    I will have a similar user input to A1, only without the moving camera. When the 
    user does something, the raytracer should redraw the scene with the block moved
    in the right way.

\subsubsection*{Add Texturing}
    I will probably just use 2D texture maps, since most of the surfaces in my final
    scene will be flat.

\subsubsection*{Final Scene}
    (Note: same text as in Statement)
    For my final scene, I hoping to make a raytraced tetris game (if I can't
    get enough FPS, I might come up with another idea for the revised proposal). The pieces
    will have different materials, instead of just different colours. I also might
    experiment with putting other primitives in the scene, like a mirror behind the
    game. It would also be interesting to make all of the pieces translucent and having
    an object behind the game to demonstrate refraction. The border of the game and the
    surface that the game sits on (probably just a plane), will have an interesting texture
    of some sort.

\item[Bibliography]:
    \bibliographystyle{alpha}

    \bibliography{references}

\end{description}
\newpage


\noindent{\Large\bf Objectives:}

{\hfill{\bf Full UserID:\rule{2in}{.1mm}}\hfill{\bf Student ID:\rule{2in}{.1mm}}\hfill}

\begin{enumerate}
     \item[\_\_\_ 1:]  Bounding Interval Hierarchy

     \item[\_\_\_ 2:]  Reflection Rays

     \item[\_\_\_ 3:]  Refraction Rays

     \item[\_\_\_ 4:]  Antialiasing

     \item[\_\_\_ 5:]  Bump Mapping

     \item[\_\_\_ 6:]  Add Tetris Primitive

     \item[\_\_\_ 7:]  Keyframe Animation with Linear Interpolation

     \item[\_\_\_ 8:]  Manage user input using Qt

     \item[\_\_\_ 9:]  Texture Mapping

     \item[\_\_\_ 10:] Final Scene
\end{enumerate}

A4 extra objective: None

\end{document}
